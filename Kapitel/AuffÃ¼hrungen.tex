\section{Vorbereitung für die Aufführungen}
\subsection{Finanzen}
Alle Ausgaben für Kostüm, Schminke, Bühnenbild und Requisiten müsst ihr selbst auslegen. Das Geld könnt ihr euch von der StuV zurückerstatten lassen, indem ihr die Kassenzettel dort abgebt. Es werden keine Privatkäufe, Kleinanzeigen sowie Käufe über Amazon finanziert.
Bei Online-Bestellungen müsst ihr neben der Rechnung auch einen Zahlungsbeleg von eurem Konto einreichen. Private Informationen können geschwärzt werden.
Die Rechnungen von Aufführungsrechten, Textbüchern, Technik und Werbung gehen direkt an die StuV und müssen nicht im Voraus von euch bezahlt werden.	

\subsection{Sponsoringverträge}
Jeder Gruppe steht es frei, sich Sponsor:innen für ihre Aufführungen zu organisieren. Sponsoringverträge können ausschließlich über die StuV abgeschlossen werden. Eigene Absprachen und Sponsoringverträge, ohne offiziellen Vertrag der Universität Würzburg, sind verboten.

\subsection{Bühnenbild}
Bei großen Bühnenbildern sowie dem Einsatz von Technik, die über das Studi(o)bühnen-Inventar hinausgeht, ist aus Brandschutz- und Sicherheitsgründen immer die Genehmigung des Hausmeisters und der Referatsleitung einzuholen. Setzt euch frühzeitig mit den entsprechenden Stellen auseinander.	

\subsection{Werbung}
\subsubsection{Allgemeines}
Jede Gruppe ist für Art und Umfang ihrer Werbung selbst verantwortlich. Mindestens muss für jedes Stück ein Plakat entworfen und gedruckt sowie unsere Instagram-Seite betrieben werden.
Das Orgateam reserviert für jedes Stück städtische Plakatständer für eine oder zwei Wochen. Die städtischen Plakatständer werden mit DIN-A2 Plakaten plakatiert, es sind pro Stück 30 Plakate fristgerecht beim Kulturamt abzugeben. 
Die Studierendenvertretung nimmt 20 DIN-A3 Plakate und verteilt diese für die Produktionen an die Fachschaften der Universität. 
Das Orgateam übernimmt den Austausch mit dem Studierendenwerk und präsentiert euer Stück auf der internen Webseite. Stellt dafür dem Orgateam euer Plakat sowie prägnante Stückinformationen digital zur Verfügung. \\

Plakate dürfen nicht willkürlich in der Stadt aufgehängt werden. Für Hauswände und Stellflächen muss immer eine schriftliche Genehmigung des Eigentürmers eingeholt werden. In Cafés und Geschäften kann nach Absprache plakatiert werden.	\\

Das Studierendenwerk genehmigt das Verteilen und Auslegen von Flyern an allen Mensen. In der Mensa am Studentenhaus gibt es außerdem einen Platz im Foyer, wo ihr ein Plakat hinhängen dürft. 
Sprecht euch mit anderen Produktionen ab, um Überschneidungen bei Flyern und Plakaten zu vermeiden und gemeinsam Werbung zu betreiben.

\subsubsection{Instagram und Twitch}
Die Instagram-Seite der Studiobühne ist die zentrale Werbeplattform während der Aufführungswoche. Beim Hochladen der Beiträge stehen jedem Stück sechs oder neun Slots zur Verfügung, es dürfen beliebig viele Storys geteilt werden, es darf pro Stück ein Highlight angelegt werden. Der Account steht euch von Beginn der Generalprobe bis Abschluss der Aufführungen zur Verfügung.
Beim Streamen eurer Aufführungen über unseren Twitch-Account ist darauf zu achten, keinen Zwischenspeicher von eurer Übertragung anzulegen, sie darf ausschließlich live verfolgbar sein. Für weitere Übertragungsrechte der Aufführungen ist vorher Rücksprache mit dem Rechtsinhaber eures Theaterstückes zu halten.	

\subsection{Sensible Inhalte und Trigger}
Jede Produktion muss spätestens zwei Wochen vor Aufführungsbeginn die eigenen sensiblen Inhalte und Triggerwarnungen definiert und an die Referatsleitung weitergeleitet haben.	
Die entsprechenden Oberkategorien der Warnhinweise finden sich auf unserer Webseite und müssen bei expliziter Darstellung von den Stücken spezifiziert und erweitert werden.
Die Warnhinweise werden ausschließlich über den QR-Code auf Programmheften und Flyern wiedergegeben und sollen nicht mittels schriftlichen oder akustischen Hinweisen während der Aufführungen erfolgen.

\subsection{Technik}
\subsubsection{Lichtpult}
Das Lichtpult ist ein Avolites Tiger Touch Pro. Dieses kann mit Version 11.4. der Titan PC Suite mit dem eigenen PV vorprogrammiert werden, um die Show danach mit USB-Stick auf das Pult zu übertragen. Online stehen eine Kurzanleitung sowie eine Reihe an Video-Tutorials für die Software zur Verfügung.	
In der Dropbox befindet sich ein Default-File, das unsere Scheinwerfer und eine 3D-Visualisierung der Stadtmensa enthält.

\subsubsection{Lichtttechnik}
Unsere Lichttechnik basiert auf LED-Leuchtmitteln. Welche Scheinwerfer zur Verfügung stehen, kann dem Patchfile in der Dropbox entnommen werden.	
Es stehen viele DMX-, Strom-, Powecon- und Kaltgerätekabel auch für dazu gemietete Technik zur Verfügung.

\subsubsection{Nebeltechnik}
In Vereinbarung mit dem Studierendenwerk darf in der Stadtmensa ausschließlich der Eurolite NH-110 Tour-Fazer mit den Fluiden Eurolite Smoke Fluid „P“ und Stairville Fast Fog für Nebeltechnik genutzt werden.\\
Der Hazer darf konstant nur auf maximal 20\% Leistung betrieben werden. Im hinteren Teil der Mensa sowie im Foyer befinden sich Rauchmelder, dieser Bereich sollte vom Haze nicht erfasst werden. Vor der Pause sowie vor dem Ende des Stückes ist gründlich zu lüften, sodass der Haze beim Verlassen des Saals nicht vom Publikum nach hinten getragen wird.	

\subsubsection{Tontechnik}
Die Tontechnik wird vom Studierendenwerk gestellt und besteht aus zwei passiven Fullrange-Lautsprechern, einem Tonrack mit Mischpult, Funkmikroempfängern und Verstärker.	
Die Studi(o)bühne ergänzt diese Grundausstattung um einige Mikrofone, Kabel, eine DI-Box und ein weiteres kleines Mischpult. 	

\subsubsection{Rigging}
Zur Befestigung von Lichttechnik stehen vom Studierendenwerk zwei Traversen zur Verfügung. Alle Scheinwerfer der Studi(o)bühne werden gebündelt zum Spielzeitbeginn aufgehängt, ohne anschließend umgehängt zu werden.	
Ausnahme bilden vier PFE-60 Scheinwerfer, die von den Produktionen selbst unter Verwendung eines Safetys aufgehängt werden können.
Im Holzverschlag links auf der Bühne (hinter dem Flügel) befindet sich die Schalttafel, mit der man die Traversen herunterlassen kann. Nur die aktuell aufführende Produktion darf die Traversen bewegen.	

\subsubsection{Strom}
Um Überlast und andere Stromprobleme zu vermeiden, sollen alle technischen Geräte sowie Geräte mit hohem Stromverbrauch an unseren Stromverteiler im linken Kabuff und nicht an reguläre Steckdosen angeschlossen werden. Am Stromverteiler sind zwei Leitungen für die vordere Traverse und eine Leitung für die hintere Traverse reserviert, sodass 3x16A explizit für die Produktionen bleiben.
Vor dem Verlassen der Mensa müssen alle technischen Geräte stromlos geschaltet werden, was durch die Hauptsicherung des Stromverteilers an der Kabuff-Wand geschieht.
Für gemietete Lichttechnik und die vier flexiblen PFE-60 Scheinwerfer steht eine Mehrfachsteckdose in der Mitte der vorderen Traverse zur Verfügung.

\subsubsection{Leiter}
Im Holzverschlag rechts auf der Bühne stehen zwei Leitern. Wann immer jemand auf der Leiter steht, muss unten mindestens ein:e Helfer:in stehen und die Leiter festhalten.

\subsubsection{Front of House}
Unter der Treppe zur Empore befindet sich eine Bodenklappe, die eine Steckdose, zwei Speakon-Anschlüsse sowie ein Multicore enthält, an welches das Tonrack des Studierendenwerks angeschlossen werden kann. Auch zwei DMX-Universen transportieren wir über dieses Multicore, die Anschlüsse sind sowohl im Tonrack als auch im Rack-Schrank im linken Kabuff markiert.	

\subsubsection{Erhalt der Technik}
Um unsere Technik zu erhalten, dürfen diese nur eingewiesene Personen aus dem Technikteam aufbauen, ändern und bedienen. Solltet ihr ohne befugte Person Zugriff auf die Technik benötigen, sprecht das zuvor mit eurer Ansprechperson aus dem Technikteam ab.	
Nach jedem Verwenden der Technik oder des Orga-Raums muss ein Bild in die Regie-Gruppe geschickt werden.	
Unsere Kabel sind mit einem orangen Tape gekennzeichnet. Die Kabel sind nicht mit anderem Equipment zu vermischen.	

\subsection{Dernière}
Nach der letzten Aufführung muss der Grundzustand sowohl im Mensasaal, auf der Bühne, als auch in den Fundi wiederhergestellt werden. Bei der Koordinierung unterstützen euch zwei Ansprechpersonen aus dem Orgateam. 	
Nachdem das Aufräumen beendet ist und die Fundi durch das Orgateam abgenommen wurden, werden diese verschlossen. Bei anschließenden Feierlichkeiten ist auf Lärmbelästigung zu achten, im Innenhof sowie auf den Balkonen darf nicht laut geschrien oder Musik abgespielt werden. 	
Alle persönlichen Gegenstände müssen nach der Derniére aus der Mensa abtransportiert worden sein. Ausnahmen sind mit der Referatsleitung abzuklären.
