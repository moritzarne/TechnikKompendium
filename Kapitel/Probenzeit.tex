\section{Die Probenzeit}
\subsection{Regelungen zur Rollenverteilung}
Die Regelung zur Rollenverteilung soll Fairness und Gleichheit zwischen allen Schauspieler:innen garantieren. 	
Innerhalb einer Spielzeit darf eine Peron nur eine große/tragende Rolle spielen. Die Regie definiert die Größe der Rolle und muss sich bei Unklarheiten an die Referatsleitung wenden.	
Die Regie-Position gilt ebenso als große/tragende Rolle. Sie darf nur in dringenden und kurzfristigen Notfällen schauspielerische Tätigkeiten in der eigenen Produktion übernehmen. Dies muss vorher mit der Referatsleitung abgesprochen sein.	
Schauspieler:innen und Regisseur:innen dürfen maximal in zwei Stücken pro Spielzeit mitwirken. 
Backstage-Personen dürfen in beliebig vielen Produktionen hinter den Kulissen mitwirken (Bühne, Technik, Maske, Orga, etc.).
Wenn Schauspielende in mehreren Produktionen mitwirken, gibt es für die Proben folgende Priorisierung unter den Stücken, die kollidierenden Produktionen müssen sich hierbei für eine optimale Verteilung absprechen:
\begin{itemize}
    \item Generalprobe
    \item Hauptprobe
    \item Wegfahrwochenende
    \item Normale Probe
    \item Hauptrolle
    \item Nebenrolle
\end{itemize}

\subsection{Orga und Regie}
Organisation und Regie sollten von unterschiedlichen Personen übernommen werden. Während die Regie die kreative Leitung der Proben und der Inszenierung übernimmt, koordiniert die Organisation die Reservierungen, Probenpläne sowie Finanzen und behält den Überblick im Backstage. 	
Für die Regie ist es unerlässlich, sich außerhalb der Proben mit Konzepten und der Inszenierung zu beschäftigen. Die Regie muss ihr Stück kennen.

\subsection{Kostenvoranschlag}
Jedes Stück muss einen Kostenvoranschlag über die geplanten Ausgaben der Produktion an die Referatsleitung schreiben. Die Frist gilt in der Winterspielzeit bis zum 15.06. und in der Sommerspielzeit bis zum 15.01..	
Die Vorlage wird den Stückverantwortlichen vom Orgateam zur Verfügung gestellt. In dieser Vorlage muss begründet werden, wofür die Ausgaben geplant sind. Das Budget bewegt sich in der Regel bei durchschnittlich 1500€. 

\subsection{Räumlichkeiten}

\subsection{Reservierungen}
\subsubsection{Studierendenwerk}
Jede Raumnutzung muss im Voraus beim Studierendenwerk angefragt und bestätigt werden. Eine entsprechende Reservierung an Herrn Bundschuh muss Folgendes beinhalten:
- Studi(o)bühne und Stückname
- Kontaktdaten der Stückverantwortlichen (Mail und Telefon)
- Probenraum inkl. Zeitraum (Mensasaal, Festsaal, etc.) 
- Mehrere Raumanfragen, für den gleichen Raum, in einer Mail bündeln, viele kleine Anfragen vermeiden.
Zugesagte Probentermine können vom Studierendenwerk wieder gestrichen werden, wenn zahlende Veranstalter die Räumlichkeiten buchen. Das Studierendenwerk informiert darüber zeitnah. Ausgenommen sind zugesagte Aufführungstermine inklusive Generalprobe.	

\subsubsection{TimeTree}
Alle Termine müssen in der gemeinsamen TimeTree Gruppe dokumentiert werden. Es wird zwischen zwei Kategorien unterschieden, welche entsprechend gekennzeichnet werden müssen:	\\
- Angefragte, aber noch nicht bestätigte Termine. Die Dokumentation soll verhindern, dass mehrere Gruppen den gleichen Termin anfragen.\\
- Bestätigte Reservierungen. Nur so kann das Orgateam die genaue Nutzung der Räumlichkeiten nachvollziehen. \\
Termine, die nicht in TimeTree dokumentiert sind, können von der Referatsleitung neu vergeben werden.

\subsection{Regeln des Studierendenwerks}
\subsubsection{Schlüssel}
Sollte ein Schlüssel verloren gehen, muss dies unverzüglich per Mail (info@swerk-wue.de) an Herrn Kampf bzw. seine Sekretärin Frau Brandl gemeldet werden, sodass sie den sperren können. 

\subsubsection{Produktionsverträge}
Die Stückverantwortlichen einer Produktion müssen zu Beginn der Spielzeit einen Produktionsvertrag unterschreiben. Dieser Vertrag verpflichtet die Unterzeichnenden, die Regeln des Studierendenwerks sowie unserer Vertragspartner:innen im eigenen Cast durchzusetzen. Bei bewusstem Verstoß haften die Stückverantwortlichen.

\subsubsection{Technikvertrag}
Alle Personen die bei der Studi(o)bühne Technik machen wollen, müssen ebenfalls einen Technikvertrag unterschreiben. Dieser Vertrag verpflichtet die Unterzeichnenden, die Regeln des Studierendenwerks sowie unserer Vertragspartner:innen zu beachten und die Regie auf Missachtung hinzuweisen. Bei bewusstem Verstoß haften neben der Regie auch die Technikverantwortlichen.

\subsection{Probenwochenende}
In der Probenzeit gibt es, neben den regulären Probenterminen, zusätzliche Probenwochenenden, welche durch das Orgateam reserviert und gleichmäßig auf die Produktionen verteilt werden. An den entsprechenden Tagen stehen Mensa und Festsaal ganztags zur Verfügung.

\subsection{Aufführungen}
Jeder Produktion stehen vier Aufführungstage sowie ein vorausgehendes Generalprobenwochenende in der Mensa zur Verfügung. 	
Um die Sicherheit und Regeneration der Stückbeteiligten zu gewährleisten, ist zur Hälfte der Aufführungen ein Pausentag einzulegen. 
Der Aufbau beginnt in der Regel um 16:00 Uhr, Einlass für das Publikum ist ab 19:30 Uhr, die Aufführungen beginnen um 20:00 Uhr.
Abweichungen sind ausschließlich mit der Referatsleitung zu besprechen.
Sobald eure Aufführungstermine bestätigt wurden, müsst ihr sie erneut an die Referatsleitung sowie an Frau Grimm und Frau Vierheilig leiten.
Hinweise für die Aufführungen:	\\
-	Der Zugang zu allen Fluchttüren muss gewährleistet sein. \\
-	Beim Bestuhlen sind alle Stühle miteinander zu verhaken.\\
-	Die Wetterautomatik ist im Stromkasten des Kabuffs auf der Bühne auszuschalten.\\
-	Die erste Reihe des Mittelblocks sollte nicht für Gäste reserviert werden, achtet darauf, nicht alle vorderen Reihen durch Reservierungen zu blockieren.\\
-	Die Toilette unter der Bühne ist während der Aufführung nicht zu benutzen, die Spülung ist im Zuschauerraum zu hören.\\
-	Es dürfen Gäste von der Empore zum Festsaal aus bei den Aufführungen zuschauen. Dieser Bereich sollte allerdings vermieden werden, da hier die Sicht stark eingeschränkt ist.\\
-	Die Kassen dürfen nur mit Absprache mit Herr Lemos, Herr Bundschuh und der Referatsleitung verschoben werden.
