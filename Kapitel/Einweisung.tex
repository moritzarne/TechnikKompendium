\section{Technische Einweisung/Grundlagen}\label{sec:einweisung}
\subsection{Grundlagen des Lichts}
Die absoluten Basics der Lichttechnik werden in den folgenden Artikeln erläutert.\\
\href{https://www.bonedo.de/artikel/crashkurs-grundlagen-der-lichttechnik-1/}{Grundlagen der Lichttechnik 1} \\
\href{https://www.bonedo.de/artikel/crashkurs-grundlagen-der-lichttechnik-2/}{Grundlagen der Lichttechnik 2} \\
\href{https://www.bonedo.de/artikel/crashkurs-grundlagen-der-lichttechnik-3/}{Grundlagen der Lichttechnik 3} 

\subsection{DMX}
DMX ist das Protokoll, über das unsere Scheinwerfer angesteuert werden. Um Fehlerquellen identifizieren zu können ist ein grobes Verständnis für DMX vorteilhaft.\\
Folgende Youtube Videos bieten einen guten Einstieg: \\
\href{https://youtu.be/w8TdsO8vqgo}{DMX Video 1}\\
\href{https://youtu.be/GMMHAhFN6LQ}{DMX Video 2}\\

\noindent Folgende Artikel bieten einen guten schriftlichen Einstieg: \\
\href{https://www.stage223.com/wissen/dmx-grundlagen-workshop-wie-funktioniert-dmx/
}{DMX Artikel 1}\\
\href{https://www.delamar.de/eventtechnik/dmx-steuerung-54218/}{DMX Artikel 2}

\subsection{Scheinwerferarten}
Um eine gute Ausleuchten zu gewährleisten, gibs es verschiedene Arten an Scheinwerfern für verschieden Zwecke. Für das Theater sind die Arten Fresnel, Profiler und Verfolger am relevantesten.\\
Hier findet ihr eine Übersicht aller möglichen Scheinwerferarten:\\
\href{https://wiki.production-partner.de/licht/welches-licht-macht-welcher-scheinwerfer/}{Scheinwerferarten Artikel}\\


\subsection{Positierung und Intention von Licht}

Der Artikel ist zwar eher auf Konzerte ausgelegt, ist aber auch durchaus für Theater interessant: \\
- Welche Intentionen kann Licht haben (Frontlicht für Sichtbarkeit, Mehrdimensionalität, Atmosphäre, Effekte…)\\
- Welche Lichtrichtungen gibt es (Frontlicht, Gegenlicht, Seitenlicht…)\\
- Unterschied der subtraktiven und additiven Farbmischung \\
- Bezüglich der Technologien zu Beachten: Wir haben in der Studiobühne nur noch LED-Scheinwerfer\\
\href{https://www.stageaid.de/komponenten-einer-lichtanlage-grundlagen/}{Scheinwerfer und Positionierung Artikel} \\


\subsection{Kabel in der Veranstaltungstechnik} 
In den folgenden Artikeln findet ihr eine Übersicht über alle möglichen Stecker und Kabel in den Bereichen Ton, Licht, Strom und Video - viele davon sind auch bei uns zu finden.  \\
\href{https://www.stage223.com/wissen/steckerarten-fuer-strom-in-der-veranstaltungstechnik-what-the-plug/}{Stecker und Kabel Strom und Licht Artikel}\\
\href{https://www.stage223.com/wissen/steckerarten-fuer-multimedia-in-der-veranstaltungstechnik-what-the-plug/}{Stecker und Kabel Multimedia Artikel} \\
\href{https://www.stage223.com/wissen/steckerarten-im-analogen-audiobereich-der-veranstaltungstechnik-what-the-plug/}{Stecker und Kabel Ton Artikel} \\


\subsection{Begrifflichkeiten} 
Folgendes PDF gibt einen Überblick über die wichtigsten Begrifflichkeiten in der Veranstaltungstechnik: \\
\href{https://bibliotheksportal.de/wp-content/uploads/2017/11/Lexikon-Veranstaltungstechnik.pdf}{Begrifflichkeiten PDF} \\


\subsection{Unser Lichtpult}
Unser Lichtpult ist ein grandMA3 Compact mit der aktuellsten Version der grandMA3 Software. Man kann sich jederzeit am eigenen PC in die Software einarbeiten:\\
Hier ist der Download für die Software:\\
\href{https://www.malighting.com/de/downloads/produkt/grandma3/}{Download Software} \\

\noindent Um unsere Fixture Types und die 3D-Visualisierung der Mensa zu nutzen, ladet euch am besten das Template-File aus dem Technik Google Drive Ordner herunter und öffnet es in der Software.\\

\noindent Besonders hilfreich für die Einarbeitung ist die MA University. Hier könnt ihr euch kostenlos anmelden und praxisnahe Tutorials im eigenen Tempo durcharbeiten. Eine Anmeldung ist erforderlich, aber vollkommen kostenfrei!\\
\href{https://www.lightpower.de/ma-university/e-learning/}{MA University}\\

\noindent Außerdem findet ihr hier viele YouTube Tutorials:\\
\href{https://www.youtube.com/playlist?list=PLhh6ZoFPnUu1hMCDT2YhuYJxLgW_G0rTn}{MA Video Tutorials}\\

\noindent Und hier findet ihr noch eine Quick Start Manual sowie das ausführliche Handbuch:\\
\href{https://help.malighting.com/grandMA3/2.3/HTML/qsg.html}{MA Quick Start Manual}\\
\href{https://help.malighting.com/grandMA3/2.3/HTML/help.html}{MA Handbuch}\\
