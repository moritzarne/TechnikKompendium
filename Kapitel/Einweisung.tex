\section{Technische Einweisung/Grundlagen}
\subsection{Grundlagen des Lichts}
- \href{https://www.bonedo.de/artikel/crashkurs-grundlagen-der-lichttechnik-1/}{Grundlagen der Lichttechnik 1} \\
- \href{https://www.bonedo.de/artikel/crashkurs-grundlagen-der-lichttechnik-2/}{Grundlagen der Lichttechnik 2} \\
- \href{https://www.bonedo.de/artikel/crashkurs-grundlagen-der-lichttechnik-3/}{Grundlagen der Lichttechnik 3} 

\subsection{DMX}
Wichtig das Protokoll zu verstehen, da so alle Scheinwerfer angesteuert werden. Auch ein Verständnis für Splitter ist gut.\\
Hier sind zwei Youtube Videos: \\
\href{https://youtu.be/w8TdsO8vqgo?feature=shared}{Video 1}\\
\href{https://youtu.be/GMMHAhFN6LQ}{Video 2}\\

\noindent Hier sind noch zwei Artikel, die das Thema zusammenfassen:\\
\href{https://www.stage223.com/wissen/dmx-grundlagen-workshop-wie-funktioniert-dmx/
}{Artikel 1}\\
\href{https://www.delamar.de/eventtechnik/dmx-steuerung-54218/}{Artikel 2}

\subsection{Scheinwerfer und Positionierung}
Der Artikel ist zwar eher auf Konzerte ausgelegt, ist aber auch durchaus für Theater interessant: \\
- Welche Intentionen kann Licht haben (Frontlicht für Sichtbarkeit, Mehrdimensionalität, Atmosphäre, Effekte…)\\
- Welche Lichtrichtungen gibt es (Frontlicht, Gegenlicht, Seitenlicht…)\\
- Unterschied der subtraktiven und additiven Farbmischung \\
-> Bezüglich der Technologien: Wir haben nur noch LED\\
\href{https://www.stageaid.de/komponenten-einer-lichtanlage-grundlagen/}{Artikel} \\

\noindent Hier findet ihr eine Übersicht aller möglichen Scheinwerferarten. Für uns besonders spannend sind Fresnel, Profiler und Verfolger.\\
\href{https://wiki.production-partner.de/licht/welches-licht-macht-welcher-scheinwerfer/}{Artikel}\\

\noindent Und in folgendem Artikel findet ihr eine Übersicht über alle möglichen Stecker und Kabel: \\
\href{https://www.stage223.com/wissen/steckerarten-fuer-strom-in-der-veranstaltungstechnik-what-the-plug/}{Artikel}

\subsection{Unser Lichtpult}
Unser Lichtpult ist ein Avolites Tiger Touch Pro. Man kann sich jederzeit am eigenen PC in die Software einarbeiten:\\
\href{https://web3.avolites.com/software/downloads/titan-pc-suite}{Software} (v11.4)\\

\noindent Gegebenenfalls müsst ihr dann noch die Fixtures Library aktualisieren:\\
\href{https://personalities.avolites.com/PersonalityFiles/Downloads/TitanFixtureLibraryV10.exe}{Fixture Libary}.\\

\noindent Hier findet ihr viele Tutorial Videos:\\
\href{https://www.avolites.com/support/tutorials-guides/}{Avolites Tutorials}\\
\href{URL}{Youtube Tutorials}\\

\noindent Und hier findet ihr noch eine Kurzanleitung als PDF: \href{https://web3.avolites.com/Portals/0/downloads/manuals/titanone/Titan%20One%20Quick%20Start%20Guide.pdf?ver=2019-06-06-150613-817}{Anleitung}.