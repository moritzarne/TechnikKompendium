\section{Technische Einweisung/Grundlagen}\label{sec:einweisung}
Die absoluten Basics der Lichttechnik werden hier erläutert.\\
\subsection{Grundlagen des Lichts}
- \href{https://www.bonedo.de/artikel/crashkurs-grundlagen-der-lichttechnik-1/}{Grundlagen der Lichttechnik 1} \\
- \href{https://www.bonedo.de/artikel/crashkurs-grundlagen-der-lichttechnik-2/}{Grundlagen der Lichttechnik 2} \\
- \href{https://www.bonedo.de/artikel/crashkurs-grundlagen-der-lichttechnik-3/}{Grundlagen der Lichttechnik 3} 

\subsection{DMX}
Wichtig das Protokoll zu verstehen, da so alle Scheinwerfer angesteuert werden. Auch ein Verständnis für Splitter ist gut.\\
Hier sind zwei Youtube Videos: \\
\href{https://youtu.be/w8TdsO8vqgo?feature=shared}{DMX Video 1}\\
\href{https://youtu.be/GMMHAhFN6LQ}{DMX Video 2}\\

\noindent Hier sind noch zwei Artikel, die das Thema zusammenfassen:\\
\href{https://www.stage223.com/wissen/dmx-grundlagen-workshop-wie-funktioniert-dmx/
}{Grundlagen Licht Artikel 1}\\
\href{https://www.delamar.de/eventtechnik/dmx-steuerung-54218/}{Grundlagen Licht Artikel 2}

\subsection{Scheinwerfer und Positionierung}
Der Artikel ist zwar eher auf Konzerte ausgelegt, ist aber auch durchaus für Theater interessant: \\
- Welche Intentionen kann Licht haben (Frontlicht für Sichtbarkeit, Mehrdimensionalität, Atmosphäre, Effekte…)\\
- Welche Lichtrichtungen gibt es (Frontlicht, Gegenlicht, Seitenlicht…)\\
- Unterschied der subtraktiven und additiven Farbmischung \\
- Bezüglich der Technologien zu Beachten: Wir haben in der Studiobühne nur noch LED-Scheinwerfer\\
\href{https://www.stageaid.de/komponenten-einer-lichtanlage-grundlagen/}{Scheinwerfer und Positionierung Artikel} \\

\noindent Hier findet ihr eine Übersicht aller möglichen Scheinwerferarten. Für uns besonders spannend sind Fresnel, Profiler und Verfolger.\\
\href{https://wiki.production-partner.de/licht/welches-licht-macht-welcher-scheinwerfer/}{Scheinwerferarten Artikel}\\

\noindent Und in folgendem Artikel findet ihr eine Übersicht über alle möglichen Stecker und Kabel: \\
\href{https://www.stage223.com/wissen/steckerarten-fuer-strom-in-der-veranstaltungstechnik-what-the-plug/}{Stecker und Kabel Artikel}

\subsection{Unser Lichtpult}
Unser Lichtpult ist ein Avolites Tiger Touch Pro. Man kann sich jederzeit am eigenen PC in die Software einarbeiten:\\
Hier ist der Download für die Software, das anshchließende Anlegen eines Kontos ist verpflichtend, aber kostenlos.\\
Da am Pult der Studiobühne eine etwas neuere Version installiert ist, könnt ihr Files vom PC auf das Pult übertragen, aber nicht umgekehrt.\\
\href{https://web3.avolites.com/software/downloads/titan-pc-suite}{Download Software} (v11.4)\\

\noindent Gegebenenfalls müsst ihr dann noch die Fixtures Library aktualisieren, um unsere Scheinwerfer in der Bibliothek zu haben:\\
\href{https://personalities.avolites.com/PersonalityFiles/Downloads/TitanFixtureLibraryV10.exe}{Download Fixture Libary}.\\

\noindent Hier findet ihr viele Tutorial Videos:\\
\href{https://www.avolites.com/support/tutorials-guides/}{Avolites Video Tutorials}\\

\noindent Und hier findet ihr noch eine Kurzanleitung als PDF:\\
\href{https://web3.avolites.com/Portals/0/downloads/manuals/titanone/Titan%20One%20Quick%20Start%20Guide.pdf?ver=2019-06-06-150613-817}{Avolites PDF Anleitung}.\\

\noindent Besonders hilfreich für die Einarbeitung ist aber die Avolites Online Learning Plattform. Hier könnt ihr euch kostenlos anmelden und praxisnahe Tutorials im eigenen Tempo durcharbeiten:\\
\href{https://www.avolitesacademy.com/online-learning/}{Avolites Online Learning}\\
