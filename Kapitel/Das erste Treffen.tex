\section{Das Erste Treffen}
\subsection{Allgemeines}
Am Anfang jedes Semesters kommt die gesamte Studi(o)bühne zum Ersten Treffen zusammen.	
Neben der Vorstellung des Theaters für neue Interessierte, werden Neuigkeiten aus dem Orgateam bekanntgegeben.
Zum Wintersemester werden die Stückvorschläge für die Sommerspielzeit vorgestellt, zum Sommersemester diejenigen für die Winterspielzeit.	
In jeder Spielzeit stehen sechs Produktionsslots zur Verfügung. Bei mehr als sechs Einreichungen werden aus den Vorschlägen werden von allen Anwesenden die Produktionen der jeweiligen Spielzeit gewählt.	
Im Anschluss besteht die Möglichkeit, sich mit den Verantwortlichen der gewählten Produktionen auszutauschen und in Castinglisten einzutragen.

\subsection{Stückeinreichung und Rechte}
Vor dem Ersten Treffen können Stückvorschläge bei der Referatsleitung eingereicht werden. Die jeweiligen Fristen sowie Informationen über Vorstellungsweise und geforderte Inhalte werden über den E-Mail Verteiler bekanntgegeben.	
In jeder Spielzeit stehen sechs Produktionsslots zur Verfügung. 
Die Rechte für eingereichte Produktionen sind frühzeitig bei den entsprechenden Verlagen anzufragen und für die Spielzeit zu reservieren. Der endgültige Erwerb der Aufführungsrechte erfolgt erst nach der Verteilung der Aufführungstermine. Die Rechnung ist an die StuV zu adressieren, tauscht euch vorher mit Frau Grimm und Frau Vierheilig aus. 

\subsection{Nachtreffen Regie}
Nach dem Ersten Treffen und den Castings, trifft sich das Orgateam mit den Stückverantwortlichen im sogenannten Nachtreffen.	
Es werden detailliert Kompendium, Proben- sowie Aufführungsslots und weitere organisatorische Punkte besprochen. Jede Produktion wird einer Ansprechperson aus dem Orgateam zugewiesen. 	
Die konkreten Themen werden den Stückverantwortlichen nach dem Ersten Treffen mitgeteilt.

\subsection{Nachtreffen Technik}
Nach dem Nachtreffen der Regie findet das Nachtreffen Technik statt. Hier wird jeder Produktion eine Person aus dem Technikteam als erste Ansprechperson zugeteilt. 
Dieser Person ist es freigestellt, ob sie nur für technische Fragen der Produktion zur Verfügung steht oder ob sie auch bei der Planung und Durchführung der Technik unterstützt.