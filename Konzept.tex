\section{Konzept}
\subsection{Idee}
Die Studi(o)bühne wurde im Wintersemester 2005/2006 von der Studierendenvertretung (StuV) gegründet.	
Ihr Ziel ist es, allen Menschen zu ermöglichen, sich mit Theater und Kultur auseinanderzusetzen. Im Rahmen der Gemeinschaftsgrundsätze unserer Charta ist jeder Mensch willkommen. Es sind keine Vorkenntnisse notwendig.
Unser Theater wird finanziell von der StuV und durch Spendeneinnahmen getragen, wodurch weder für unsere Mitwirkenden, noch für das Publikum finanzielle Barrieren entstehen. 	
Die Mitwirkenden auf und hinter der Bühne organisieren sich in ihren Theaterproduktionen selbstständig. 

\subsection{Studierendenvertretung}
Die wichtigsten Organe der StuV sind die Fachschaftsvertretungen, der Fachschaftenrat, das Studierendenparlament, sowie der Studentische Sprecher:innenrat. Zusätzlich gibt es verschiedene Referate, mit unterschiedlichen Themen, zum Beispiel die Referate Studi(o)bühne, Queer:feminismus oder Referat Demokratie.
Die wichtigsten Ansprechpersonen sind Pia Grimm und Christine Vierheilig.

\subsection{Studierendenwerk}
Das Studierendenwerk Würzburg stellt den Großteil unserer Räumlichkeiten für Lagerung, Proben sowie Aufführungen.	
Die Studi(o)bühne ist verpflichtet, während der Aufführungen Getränke des Studierendenwerks zu verkaufen.  
Die wichtigsten Ansprechpersonen sind Dominik Kampf (Geschäfts- führung), Andreas Bundschuh (Catering und Raumbelegung), Tanja Scheller (Öffentlichkeitsarbeit), Sebastian Lemos (Hausmeister).

\subsection{Referatsleitung}
Die Referatsleitung ist hauptverantwortlich für die Studi(o)bühne. Sie wird durch das Studierendenparlament für ein Jahr gewählt.
Aufgabe ist die enge Kommunikation und Vermittlung zwischen der Studi(o)bühne und ihren Vertragspartnern sowie der StuV. Es müssen ein jährlicher Bericht über die Aktivitäten sowie die Haushaltsplanung des Referats bei der StuV eingereicht werden.	
Weiterhin ist die Referatsleitung verantwortlich für die rechtliche Absicherung, transparente Darstellung und Optimierung interner Abläufe.
Konkrete Verpflichtungen und Regelungen für die Referatsleitung sind der Charta der Studi(o)bühne zu entnehmen.	

\subsection{Orgateam}
Das Orgateam besteht aus Mitgliedern der Studiobühne, welche von der Referatsleitung ausgewählt werden. Es unterstützt die Referatsleitung bei anfallenden organisatorischen Aufgaben.	 
Das Orgateam setzt sich zusammen aus:\\
Moritz Wübbena ist Referatsleitung.\\
Änni Oberdorfer ist Referatsleitung.\\
Isabel Schultz ist Ansprechperson bei Fragen zur Raumreservierung. Sie verwaltet unseren E-Mail-Verteiler und ist für die Reservierung der blauen Plakatständer des Kulturamtes zuständig.\\
Nex Arabschahi ist Fundus- und Maskenbeauftragte Person.\\
Julian Schmidt ist hauptverantwortlich für alle technischen Fragen und koordiniert das Technikteam. \\
Kathi Schricker ist zuständig für das Merchandise sowie für den Internetauftritt.\\
Catharina Hamann ist zuständig für Externe Kommunikation. 

\subsection{Technikteam}
Das Technikteam besteht aus Expert:innen zu Technikfragen. Jeder Produktion wird eine Person aus dem Technikteam als erste Anlaufstelle und feste Ansprechperson bei Technikfragen zugeteilt.	