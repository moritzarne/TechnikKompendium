\section{Vorbereitung für die Aufführungen}
\subsection{Finanzen}
Alle Ausgaben für Kostüm, Schminke, Bühnenbild, Requisiten und Technik müsst ihr selbst auslegen. Das Geld könnt ihr euch von der StuV zurückerstatten lassen, indem ihr die Kassenzettel dort abgebt. Es werden keine Privatkäufe, Kleinanzeigen sowie Käufe über Amazon finanziert.
Bei Online-Bestellungen müsst ihr neben der Rechnung auch einen Zahlungsbeleg von eurem Konto einreichen. Private Informationen können geschwärzt werden.
Die Rechnungen von Aufführungsrechten, Textbüchern, Technik von Schenkspass und Werbung gehen direkt an die StuV und müssen nicht im Voraus von euch bezahlt werden.	

\subsection{Sponsoringverträge}
Jeder Gruppe steht es frei, sich Sponsor:innen für ihre Aufführungen zu organisieren. Sponsoringverträge können ausschließlich über die StuV abgeschlossen werden. Eigene Absprachen und Sponsoringverträge, ohne offiziellen Vertrag der Universität Würzburg, sind verboten.

\subsection{Bühnenbild}
Bei großen Bühnenbildern sowie dem Einsatz von Technik, die über das Studi(o)bühnen-Inventar hinausgeht, ist aus Brandschutz- und Sicherheitsgründen immer die Genehmigung des Hausmeisters und der Technischen Leitung einzuholen. Setzt euch frühzeitig mit den entsprechenden Stellen auseinander.	

\subsection{Twitch}
Beim Streamen eurer Aufführungen über unseren Twitch-Account ist darauf zu achten, keinen Zwischenspeicher von eurer Übertragung anzulegen, sie darf ausschließlich live verfolgbar sein. Für weitere Übertragungsrechte der Aufführungen ist vorher Rücksprache mit dem Rechtsinhaber eures Theaterstückes zu halten. 

\subsection{Sensible Inhalte und Trigger}
Jede Produktion muss spätestens zwei Wochen vor Aufführungsbeginn die eigenen sensiblen Inhalte und Triggerwarnungen definiert und an die Referatsleitung weitergeleitet haben.	
Die entsprechenden Oberkategorien der Warnhinweise finden sich auf unserer Webseite und müssen bei expliziter Darstellung von den Stücken spezifiziert und erweitert werden.
Die Warnhinweise werden ausschließlich über den QR-Code auf Programmheften und Flyern wiedergegeben und sollen nicht mittels schriftlichen oder akustischen Hinweisen während der Aufführungen erfolgen.
Denkt daran, dass auch Technik Trigger enthalten kann, wie z.B. laute Geräusche, Stroboskoplicht oder Nebel.

\subsection{Dernière}
Nach der letzten Aufführung muss der Grundzustand sowohl im Mensasaal, auf der Bühne, als auch in den Fundi wiederhergestellt werden. Bei der Koordinierung unterstützen euch zwei Ansprechpersonen aus dem Orgateam. 	
Nachdem das Aufräumen beendet ist und die Fundi durch das Orgateam abgenommen wurden, werden diese verschlossen. Bei anschließenden Feierlichkeiten ist auf Lärmbelästigung zu achten, im Innenhof sowie auf den Balkonen darf nicht laut geschrien oder Musik abgespielt werden. 	
Alle persönlichen Gegenstände müssen nach der Derniére aus der Mensa abtransportiert worden sein. Ausnahmen sind mit der Referatsleitung abzuklären.
