\section{Technik}
\subsection{Lichtpult}
Das Lichtpult ist ein grandMA3 Compact mit der aktuellsten Version der grandMA3 Software. \\
Die Software ist auch für MacOS und Windows kostenlos verfügbar, um auf dem eigenen Rechner Shows vor zu programmieren. 
Die Show kann anschließend mit einem USB-Stick auf das Pult übertragen werden.\\
Online stehen eine Kurzanleitung, die MA University sowie eine Reihe an Video-Tutorials für die Software kostenlos zur Verfügung (siehe \ref{sec:einweisung}).\\
Im geteilten Technik Google Drive Ordner befindet sich ein Template-File, das sowohl am Pult als Basis als auch zum Vorprogrammieren verwendet werden sollte. Darin sind unsere Scheinwerfer (Fixture Types) sowie eine 3D-Visualisierung der Stadtmensa enthalten.

\subsection{Lichttechnik}
Unsere Lichttechnik basiert auf LED-Leuchtmitteln. Welche Scheinwerfer zur Verfügung stehen, kann dem Patchfile in dem Google Drive Ordner entnommen werden.	
Es stehen viele DMX-, Strom-, Powercon- und Kaltgerätekabel auch für dazu gemietete Technik zur Verfügung.

\subsection{Nebeltechnik}
In Vereinbarung mit dem Studierendenwerk darf in der Stadtmensa ausschließlich der Eurolite NH-110 Tour-Fazer mit den Fluiden Eurolite Smoke Fluid „P“ und Stairville Fast Fog für Nebeltechnik genutzt werden.\\
Der Hazer darf konstant nur auf maximal 20\% Leistung betrieben werden, nur für weniger als 3 Minuten ist die volle Leistung erlaubt. \\
Im hinteren Teil der Mensa sowie im Foyer befinden sich Rauchmelder, dieser Bereich sollte vom Haze nicht erfasst werden. Vor der Pause sowie vor dem Ende des Stückes ist gründlich zu lüften, sodass der Haze beim Verlassen des Saals nicht vom Publikum nach hinten getragen wird.	

\subsection{Tontechnik}
Die Tontechnik wird vom Studierendenwerk gestellt und besteht aus zwei passiven Fullrange-Lautsprechern, einem Tonrack mit Mischpult, Funkmikroempfängern und Verstärker.	
Die Studi(o)bühne ergänzt diese Grundausstattung um einige Mikrofone, Kabel, eine DI-Box und ein weiteres kleines Mischpult. 
TODO

\subsection{Rigging}
Zur Befestigung von Lichttechnik stehen vom Studierendenwerk zwei Traversen zur Verfügung. Alle Scheinwerfer der Studi(o)bühne werden gebündelt zum Spielzeitbeginn aufgehängt, ohne anschließend umgehängt zu werden.	
Ausnahme bilden vier PFE-60 und zwei COB-200 Scheinwerfer, die von den Produktionen selbst unter Verwendung eines Safetys aufgehängt werden können.
Im Holzverschlag links auf der Bühne (hinter dem Flügel) befindet sich die Schalttafel, mit der man die Traversen herunterlassen kann. Nur die aktuell aufführende Produktion darf die Traversen bewegen.\\
Es ist stets darauf zu achten, dass die Traversen nicht überlastet werden und alle Scheinwerfer mit einem passend dimensionierten Safety gesichert sind.	

\subsection{Strom}
Aus Sicherheitsgründen müssen alle für die Aufführung verwendeten technischen Geräte an unseren Stromverteiler im linken Kabuff und nicht an reguläre Steckdosen angeschlossen werden. \\ 
Am Stromverteiler sind zwei Leitungen für die vordere Traverse und eine Leitung für die hintere Traverse reserviert, sodass 2x16A explizit für die Produktionen bleiben.\\
Vor dem Verlassen der Mensa müssen alle technischen Geräte stromlos geschaltet werden, was durch die Hauptsicherung des Stromverteilers an der Kabuff-Wand geschieht.
Für gemietete Lichttechnik und die vier flexiblen PFE-60 Scheinwerfer steht eine Mehrfachsteckdose in der Mitte der vorderen Traverse zur Verfügung.

\subsection{Leiter}
Im Holzverschlag rechts auf der Bühne stehen zwei Leitern. Wann immer jemand auf der Leiter steht, muss unten mindestens ein:e Helfer:in stehen und die Leiter festhalten.

\subsection{Erhalt der Technik}
Um unsere Technik zu erhalten, dürfen diese nur eingewiesene Personen aus dem Technikteam aufbauen, ändern und bedienen. Dritten Personen ist der Umgang mit der Technik untersagt. \\
Nach jedem Verwenden der Technik oder des Orga-Raums muss ein Bild in die Regie-Gruppe geschickt werden.	\\
Unsere Kabel sind mit einem orangen Tape sowie einer farblichen Längenmarkierung gekennzeichnet. Geräte sind mit einem StuV-Sticker versehen. Unser Equipment, ins Besondere die Kabel, sind nicht mit externem Equipment zu vermischen.

\subsection{Technik Kabuff}
Um den Auf- und Abbau so einfach wie möglich zu gestalten, ist ein großer Teil unserer Steuerungslogik fest im Kabuff vorne links in einem Serverschrank verbaut. Im Aufführungsbetrieb ist dieser offen und frei zugänglich. Zwischen den Spielzeiten oder bei längeren Spielpausen ist der Schrank verschlossen. Der Schlüssel ist im Technik-/Orgaraum im Mäusefundus unter der Bühne zu finden in einem Schlüsseltresor. Der aktuelle PIN für diesen Tresor ist der WhatsApp Gruppenbeschreibung zu entnehmen und sollte nicht frei geteilt werden.
Der Vollständigkeit halber wird hier die gesamte verbaute Technik beleuchtet. Alles \color{purple} Lila \color{black} hinterlegte übersteigt aber unsere Verwendung im normalen Theaterbetrieb und kann daher übersprungen werden.

\subsubsection{Komponenten}

Zunächst werden die einzelnen Komponenten beleuchtet, bevor konkrete Setups zum Betrieb vorgestellt werden
\\\\
\noindent\textbf{Bodenklappe}\\
\includegraphics[width=0.5\textwidth]{graphics/bodenklappe.jpeg}\\
Unter der Treppe zur Empore ist eine Bodenklappe, die alle notwendigen Kabel enthält um die Technik von hier zu betreiben. Das sind
\begin{itemize}
    \item Zwei CAT-Kabel (RJ45), beschriftet mit CAT-1 (für Licht) und CAT-2 (für Ton)
    \item Ein Schuko-Dose für Strom
    \color{purple}
    \item Zwei Speakon Kabel, die verwendet werden können, falls das Roll-Rack nach hinten geschoben wird
    \item Ein Harting Buchse, der verwendet werden kann, falls das Roll-Rack nach hinten geschoben wird
\end{itemize}
Zur Verlängerung gibt es ein Kombikabel mit einem Schuko- und zwei CAT-Kabeln gebündelt. 
Weitere Infos zum Anschluss werden in den folgenden Kapiteln gegeben.
\\\\
\noindent\textbf{Grauer Rack Schrank}\\
% TODO Bild grauer Rack Schrank mit BEschriftung der Bestandteile
% \includegraphics[width=0.5\textwidth]{graphics/grauer_rack_schrank.jpeg}\\ 
\color{purple}
Die im Rack Schrank verbaute Technik ist Eigentum des Studierendenwerks und beinhaltet insbesondere die Verstärker für die nach hinten versetzten Lautsprecher im Mensasaal. Der Stromschalter am oberen Ende ist daher auch nur anzuschalten, wenn diese Saal-PA verwendet wird, was bei unseren Aufführungen im Normalfall nicht zutrifft. \\
Außerdem ist unten noch eine Harting Buchse sowie einige XRL Buchsen verbaut. Intern sind die XLR Buchen direkt mit dem Multicore verbunden. \\
\color{black}
Ebenfalls verbaut sind zwei Speakon Buchsen, die intern mit den Lautsprechern verkabelt sind. Weitere Informationen zur Verkabelung finden sich im folgenden Abschnitt Roll-Rack.
\\\\
\noindent\textbf{Roll-Rack}\\
% TODO neues Bild rollendes Roll-Rack mit Rollen von Hinten (Verstärker) ohne Sub, Beschriftung Speakon Buchsen und XLR Inputs.
% \includegraphics[width=0.5\textwidth]{graphics/rollendes_roll_rack_mit_rollen.jpeg}\\ 
Das rollende Roll-Rack mit Rollen ist ebenfalls Eigentum des Studierendenwerks. Die wichtigsten verbauten Komponenten sind zwei Funkmikro-Empfänger, der Verstärker und das Mischpult. \\
\color{purple}
Das Rack hat zum Anschluss einen Harting-Stecker, der im Rack alle Ein- und Ausgänge auf XLR Stecker verteilt, welche ans Mischpult angeschlossen werden können. Der Harting-Stecker kann entweder im Kabuff mit dem grauen Rack Schrank verbunden werden oder  mit einer gespiegelten Buchse in der Bodenklappe unter der Treppe zur Empore, und löst seine Ein/Ausgänge in den XLR-Buchsen im Rack-Schrank auf. Mikrofone oder Monitore können also an diese XLR-Buchsen im Grauen Rack-Schrank angeschlossen werden, und das Roll-Rack vom Studierendenwerk dann wiederum an die Harting Buchse vorne oder hinten. \\
\color{black}
Außerdem müssen die Ausgänge des Verstärkers mit den Speakon Buchsen entweder im grauen Rack Schrank oder in der Bodenklappe unter der Empore mit den zugehörigen Speakon Buchsen verbunden werden. Diese sind intern beide gespiegelt mit den Lautsprechern verkabelt, sodass sowohl von vorne als auch von hinten Signal eingespeist werden kann. Für die Verkabelung liegen gebündelte 3m bzw. 10m lange Speokon Kabel (Eigentum Studiobühne) bereit.
\\\\
\noindent\textbf{Server-Schrank}\\
% TODO Komponenten beschriften
\includegraphics[width=0.5\textwidth]{graphics/server_schrank.jpeg}\\
Der Serverschrank ist Eigentum der Studiobühne und Herz unsere Technik im normalen Betrieb.
Hier finden sich, von oben nach unten:
\begin{itemize}
    \item WLAN-Router (auf dem Rack)
    \item 8x AA-Akkus und Akku Ladegerät für die Funkmikros
    \item ArtNet/sACN-Node 
    \item CAT-XLR-Verteiler für Ton
    \item Netzwerk-Switch für Licht und zusätzlichen Netzwerk-Traffik (z.B. Video)
    \item Kabeldurchführung
    \item DMX-Splitter
    \item Stromverteiler
\end{itemize}

Die Komponenten im Detail:
\begin{itemize}
    \item \textbf{Strom: }Der Stromverteiler sorgt für unseren eigenen abgesicherten Stromkreis. Zwei Ausgänge gehen an die vordere Traverse, ein Ausgang an die hintere Traverse und ein Ausgang versorgt das restliche Rack. Zwei weitere Ausgänge werden unten am Rack als Schuko-Buchsen herausgeführt. Diese können für stückspezifische Setups verwendet werden. Jegliche Technik eines Stücks muss an diesem Verteiler angeschlossen werden, sodass vor und nach Vorstellungen nur die Hauptsicherung an/ausgeschaltet werden muss und in unserem Stromkreis abgesichert ist.
    \item \textbf{Ton: }Der CAT-XLR-Verteiler (Pronomic NetCore System) nutzt ein Netzwerk-Kabel physikalisch zur Übertragung von bis zu 4 XLR-Signalen, indem die Pins entsprechend belegt sind. Intern werden die XLR-Buchsen auf die 16 Adern eines CAT-Kabels umgelötet. Somit findet keine Digitalisierung des Signal statt. Für jeden der 4 Kanäle kann entweder Input ODER Output verwendet werden. Es steht dann eine entsprechende Stagebox (Gegenstück) bereit, dass im Saal verwendet werden kann. Das CAT-2 Kabel verbindet die Stagebox direkt mit dem im Rack verbauten CAT-XLR-Verteiler.\\
    Die Ausgänge 1 und 2 sind standardmäßig schon mit 2 XLR Kabeln verbunden, die über dem grauen Rack enden und dort je nach Bedarf verkabelt werden können.
\end{itemize}
\includegraphics[width=0.5\textwidth]{graphics/main_cable.jpeg}\\
\includegraphics[width=0.5\textwidth]{graphics/stagebox.jpg}\\
\begin{itemize}
    \item \textbf{Licht: }Über die CAT-1 Leitung erreichen die ArtNet/sACN Pakete unseren Netzwerk Switch im Rack. An diesem ist Switch ist zum einen ein WLAN-Router angeschlossen als auch ein ArtNet/sACN Node. Dieser kann sowohl ArtNet- als auch sACN Pakete in DMX Signale umwandeln (siehe Kapitel Lichtsteuerung). Das DMX Signal wird anschließend im DMX Splitter auf 8 identische Ausgänge gespiegelt. Zwei führen fest in die vorderer Traverse und eines in die hintere Traverse. Die anderen 5 DMX-Ausgänge können von Produktionen selbst verwendet werden, z.B. für den Anschluss des Hazers oder geliehene Schenkspass Technik.\\
    \color{purple}
    Da ArtNet (max. 300 Mbit/s) nicht die volle Bandbreite des Netzwerkes (max. 1 Gbit/s) in Anspruch nimmt, kann an den Netzwerkswitch und damit an den Router auch noch eigene Technik oder Video Streams angeschlossen werden.
    \color{black}
\end{itemize}


\subsubsection{Betrieb}

\noindent\textbf{Lichtsteuerung}\\
Die Lichtsteuerung wird immer gleich verkabelt. Das Lichtpult gibt am Netzwerk Interface 1 das ArtNet Signal aus (ArtNet ist ein IP basiertes Internetprotokoll, dass die DMX Signale in IP Pakete über Netzwerk überträgt). \\
Das Interface 1 wird immer an CAT-1 aus der Bodenklappe angeschlossen (\textbf{NIEMALS CAT-2}). \\
\color{purple}
Bei Bedarf kann zwischen Pult und CAT-1 Kabel noch ein Netzwerkswitch gepatcht werden, um eigene Geräte oder Videotechnik zu verbinden.
\color{black}
\\\\
\noindent\textbf{Ton}\\
Bei Ton gibt es verschiedene Verkabelungsoptionen, je nach den Anforderungen der Produktionen:\\
\begin{itemize}
    \item \textbf{Nur ein Laptop für Musik, Sounds etc benötigt:}\\
    In diesem einfachsten Szenario wird der Laptop an eine DI-Boxen angeschlossen, diese wird dann an die Eingänge 1 und 2 der CAT-Stagebox angeschlossen, die mit CAT-2 in der Bodenklappe verbunden ist. Im Kabuff sind an die Ausgänge 1 und 2 des CAT-XLR-Verteilers wie bereits beschrieben schon zwei XLR angeschlossen, die dann neben dem grauen Rack Schrank von der Decke hängen. Diese werden in zwei Eingänge des Mischpults im Roll-Rack angeschlossen. Die beiden XLR-Ausgänge des Mischpults werden mit den Eingängen des Verstärker verbunden, die Speakon-Ausgänge des Verstärker mit dem Speakon-Buchsen des grauen Rack-Schranks.\\
    Nachteil dieser Lösung ist, dass am FoH keine Fader zur Verfügung stehen. Meistens können Sounds/Musik aber auch problemlos direkt am Laptop eingepegelt werden.  
    \item \textbf{kleines Mischpult hinten:}\\
    Ist von hinten eine feinere Lautstärkeregelung notwendig oder sollen noch weitere Komponenten angeschlossen werden, kann vom Rechenzentrum der Uni ein kleines Mischpult gemietet werden. Die Ausgänge dieses Mischpult werden dann an die Eingänge 1 und 2 der Netzwerk Stagebox angeschlossen (CAt-2).  Im Kabuff sind an die Ausgänge 1 und 2 des CAT-XLR-Verteilers wie bereits beschrieben schon zwei XLR angeschlossen, die dann neben dem grauen Rack Schrank von der Decke hängen. Diese werden in diesem Modi direkt in die Eingänge des Verstärkers gesteckt, und die Speakon-Ausgänge des Verstärker mit dem Speakon-Buchsen des grauen Rack-Schranks verbunden.
    \item \textbf{großes Mischpult hinten:}\\
    Alternativ kann das gesamte Roll-Rack nach hinten geschoben werden. Alle Eingänge können je nach Bedarf an das Mischpult des Roll-Racks angeschlossen werden, die Ausgänge des Mischpults entsprechend an den Verstärker. Die Speakon Ausgänge des Verstärkers können dann mit den gebündelten 10m Speakon Kabeln an die Speakon-Buchsen in der Bodenklappe angeschlossen werden. Das CAT-2 ist in diesem Fall gar nicht in Verwendung.\\
    Falls es zu Ton-Problemen kommen sollte, ist das meistens die solideste Not-Lösung.
\end{itemize}